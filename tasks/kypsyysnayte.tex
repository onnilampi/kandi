\documentclass[a4paper, 12pt, finnish]{report}
\usepackage[utf8]{inputenc}
\usepackage{amsfonts}
\usepackage{graphicx}
\usepackage[finnish]{babel}
\usepackage[nottoc,numbib]{tocbibind}
\usepackage{titlesec}
\titleformat{\chapter}
  {\Large\bfseries}
    {}            
      {0pt}      
        {\huge} 
        \titlespacing*{\chapter}{0pt}{-50pt}{40pt}
\usepackage{hyperref}
\hypersetup{pdfpagemode=UseNone, pdfstartview=FitH, colorlinks=true,urlcolor=red,linkcolor=blue,citecolor=black,pdftitle={},pdfauthor={}}
\setlength{\parindent}{0mm}
\setlength{\emergencystretch}{15pt}
\renewcommand{\bibname}{References}
\newcommand*{\findate}{\the\day.\the\month.\the\year} 

\begin{document}



\chapter{Yhteenveto}
Kuormantasaus www-palvelimissa\\
Kypsyysnäyte- 7.8.2017\\
Onni Lampi - 430023

\section*{}

Kuormantasaus www-palvelimissa on nykypäivän Internet-infrastruktuurin kannalta yksi tärkeimmistä tutkimusaiheista.
Kuormantasaus voidaan toteuttaa monella teknisesti toisistaan täysin eroavilla tavoilla, eikä yhtä oikeaa tapaa toimia välttämättä ole.
Työn tarkoituksena ja perimmäisenä motivaationa oli vertailla eri teknologioita keskenään.\\

Internetin alkuajoista lähtien vuosikymmenten ajan vallitsi tarjotuissa palveluissa selkeä trendi: yksi palvelu, yksi palvelin.
Palvelinten resurssit ja kyky vastata pyyntoihin eivät kuitenkaan skaalaudu lineaarisesti, vaan vastaan tulee usein jo pelkkä kaistanleveys.
Internetin laajentuessa ja kasvaessa nousee hajautettujen palvelinratkaisujen merkitys yhä suuremmaksi;
käyttäjän kannalta on yhdentekevää, mistä sisältö jaellaan.
Tärkeintä on saada sisältö toimitettua loppukäyttäjälle luotettavasti ja nopeasti.
Mitä enemmän edellä mainittuja ratkaisuja käytetään tuotantoympäristöissä, sitä merkityksellisempää on kuormantasaus.
Kuormantasauksella voidaan saavuttaa merkittäviä ja konkreettisia etuja kaistanleveyden ja palvelinkapasiteetin säästämisessä.\\

Tässä työssä käsiteltiin ja vertailtiin erilaisia kuormantasaustenkiikoita, niiden skaalautuvuutta sekä muita ominaisuuksia.
Työssä keskityttiin muutamaan jo olemassa olevaan ja laajasti käytettiin teknologiaan.\\

Kuormantasausta voidaan harjoittaa monin eri tavoin, toisistaan jopa merkittävästi eroavilla tekniikoilla.
Karkeasti ajatellen kuormantasaustekniikat jakautuvat kolmeen erilaiseen leiriin: sijaintiperusteisiin,
sisällön huomioon ottaviin ja sisällöstä piittaamattomiin tekniikoihin.\\

Sijaintiperusteisessa ratkaisussa sisältö palvellaan käyttäjälle parhaan käytössä olevan yhteyden ylitse.
Esimerkiksi Yhdysvalloissa ja Saksassa voidaan ylläpitää omia palvelimiaan, joista toinen palvelee Amerikan mannerta ja toinen Euraasiaa.
Palvelun laadussa saavutetaan merkittävä parannus, sillä viive tiedonsiirrossa vähenee merkittävästi jo pelkästään maantieteellisten seikkojen takia.
Sijaintiperusteinen kuormantasaus on tekniikkana suhteellisen helposti toteutettava ja ei ääritapauksessa tarvitse lainkaan älyä taakseen maantieteellisen sijainnin kartoittamisen lisäksi.\\

Sisältöperusteiset ratkaisut ottavat huomioon käyttäjän ja palvelun tarpeet.
Esimerkiksi vaikka jonkin yrityksen staattinen informaatiosisältö saatetaan palvella verkkosivuilta nopeasta välimuistista,
resurssi-intensiivisen ja harvinaisemman verkkokauppapalvelun palvelee toinen käyttöön optimoitu palvelin.\\

Kolmas karkea kategoria on sisällöstä riippumattomat teknologiat.
Näitä ei tule sekoittaa sijaintiperusteisiin tekniikoihin, vaan kyseessä on vieläkin koneellisempi ratkaisu.
Kuormaa tasataan esimerkiksi tiukalla jonoperiaatteella, esimerkiksi kolme palvelinta kukin vuorollaan ottavat yhden käyttäjän palveltavakseen.\\
	
Tämän työn tavoitteena oli perehtyä ja oppia ymmärtämään erilaisia kuormantasaustekniikoita, sekä selventää kirjallisuudessa käsitellyt merkittävät erot erilaisten teknologioiden välillä.
Keskiössä oli myös selvitys siitä, miksi erilaisissa tuotantoympäristöissä käytetään tiettyjä teknologioita.\\
	

\end{document}
