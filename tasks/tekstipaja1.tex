\documentclass[a4paper, 12pt, finnish]{report}
\usepackage[utf8]{inputenc}
\usepackage{amsfonts}
\usepackage{graphicx}
\usepackage[finnish]{babel}
\usepackage[nottoc,numbib]{tocbibind}
\usepackage{titlesec}
\titleformat{\chapter}
  {\Large\bfseries}
    {}            
      {0pt}      
        {\huge} 
        \titlespacing*{\chapter}{0pt}{-50pt}{40pt}
\usepackage{hyperref}
\hypersetup{pdfpagemode=UseNone, pdfstartview=FitH, colorlinks=true,urlcolor=red,linkcolor=blue,citecolor=black,pdftitle={},pdfauthor={}}
\setlength{\parindent}{0mm}
\setlength{\emergencystretch}{15pt}
\renewcommand{\bibname}{References}
\newcommand*{\findate}{\the\day.\the\month.\the\year} 

\begin{document}



\chapter{Yhteenveto}
Kuormantasaus www-palvelimissa\\
1. tekstipaja - 15.6.2017\\
Onni Lampi - 430023

\section*{}

Internetin alkuajoista lähtien vuosikymmenten ajan vallitsi tarjotuissa palveluissa selkeä trendi: yksi palvelu, yksi palvelin.
Palvelinten resurssit ja pyyntoihin vastaamiskyky eivät kuitenkaan skaalaudu lineaarisesti, vaan vastaan tulee usein jo pelkkä kaistanleveys.
Internetin laajentuessa ja kasvaessa nousee hajautettujen palvelinratkaisujen merkitys yhä suuremmaksi.
Mitä enemmän edellä mainittuja ratkaisuja käytetään tuotantoympäristöissä, sitä merkityksellisempää on kuormantasaus.
Kuormantasauksella voidaan saavuttaa merkittäviä ja konkreettisia etuja kaistanleveyden ja palvelinkapasiteetin säästämisessä.\\

Kuormantasausta voidaan harjoittaa monin eri tavoin, toisistaan jopa merkittävästi eroavilla tekniikoilla.
Karkeasti ajatellen kuormantasaustekniikat jakautuvat kolmeen erilaiseen leiriin: sijaintiperusteisiin, sisällön huomioon ottaviin ja sisällöstä piittaamattomiin tekniikoihin.
Tässä työssä käsitellään ja vertaillaan erilaisia kuormantasaustenkiikoita, niiden skaalautuvuutta sekä muita ominaisuuksia.
Työssä keskitytään muutamaan jo olemassa olevaan ja laajasti käytettiin teknologiaan.
Lisäksi käsitellään nopeasti mahdollisesti uusia ja kokeellisia teknologioita.\\
	
Tämän työn tavoitteena on perehtyä ja oppia ymmärtämään erilaisia kuormantasaustekniikoita, sekä selventää kirjallisuudessa käsitellyt merkittävät erot erilaisten teknologioiden välillä.
Keskiössä on myös selvitys siitä, miksi erilaisissa tuotantoympäristöissä käytetään tiettyjä teknologioita.
Tarkastelun ei ole tarkoitus pureutua jokaisen teknologian historiaan ja totuttuihin käytänteisiin, vaan tutkia aihetta juurikin käyttöasteen kannalta.\\
	

\end{document}
