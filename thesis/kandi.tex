%%%%%%%%%%%%%%%%%%%%%%%%%%%%%%%%%%%%%%%%%%%%%%%%%%%%%%%%%%%%%%%%%%%%
%%%%%%%%%%%%%%%%%%%%%%%%%%%%%%%%%%%%%%%%%%%%%%%%%%%%%%%%%%%%%%%%%%%%
%%                                                                %%
%% Esimerkki opinnäytteen tekemisestä LaTeX:lla                   %%
%% Alkuperäinen versio Luis Costa,  muutokset Perttu Puska        %%
%% Ruotsinkielen tuki lisätty 15092014                            %%
%%                                                                %%
%% Tähän esimerkkiin kuuluu tiedostot                             %%
%%         opinnaytepohja.tex (versio 2.01)                       %%
%%         thesistemplate.tex (versio 2.01) (for text inEnglish)  %%
%%         aaltothesis.cls (versio 2.01)                          %%
%%         kuva1.eps                                              %%
%%         kuva2.eps                                              %%
%%         kuva1.pdf                                              %%
%%         kuva2.pdf                                              %%
%%                                                                %%
%%                                                                %%
%% Kääntäminen joko                                               %%
%% latex:                                                         %%
%%             $ latex opinnaytepohja                             %%
%%             $ latex opinnaytepohja                             %%
%%                                                                %%
%%   Tuloksena on tiedosto opinnayte.dvi, joka                    %%
%%   muutetaan ps-muotoon seuraavasti                             %%
%%                                                                %%
%%             $ dvips opinnaytepohja -o                          %%
%%                                                                %%
%%   ja edelleen pdf-muotoon seuraavasti                          %%
%%                                                                %%
%%             $ ps2pdf opinnaytepohja.ps                         %%
%%                                                                %%
%% Tai                                                            %%
%% pdflatex:                                                      %%
%%             $ pdflatex opinnaytepohja                          %%
%%             $ pdflatex opinnaytepohja                          %%
%%                                                                %%
%%   Tuloksena on tiedosto opinnaytepohja.pdf                     %%
%%                                                                %%
%% Selittävät kommentit on tässä esimerkissä varustettu           %%
%% %%-merkeillä ja muutokset, joita käyttäjä voi tehdä,           %%
%% on varustettu %-merkeillä                                      %%
%%                                                                %%
%%%%%%%%%%%%%%%%%%%%%%%%%%%%%%%%%%%%%%%%%%%%%%%%%%%%%%%%%%%%%%%%%%%%
%%%%%%%%%%%%%%%%%%%%%%%%%%%%%%%%%%%%%%%%%%%%%%%%%%%%%%%%%%%%%%%%%%%%

%% Käytä toinen näistä:
%% ensimmäinen, jos käytät pdflatexia, joka kääntää tekstin suoraan 
%% pdf-tiedostoksi (kuvat on oltava jpg- tai pdf-tiedostoina)
%% toinen, jos haluat tuottaa ps-tiedostoa (käytä eps-formaattia kuville,
%% alä käytä ps-muotoisia kuvia!)
%%
\documentclass[finnish,12pt,a4paper,pdftex,elec,utf8]{aaltothesis}
%\documentclass[finnish,12pt,a4paper,dvips]{aaltothesis}

%% Kirjoita y.o. \documentclass optioiksi
%% korkeakoulusi näistä: arts, biz, chem, elec, eng, sci
%% editorisi käyttämä merkkikoodaustapa: utf8, latin1
%%

%% Käytä näitä, jos kirjoitat englanniksi. Katso englanninokset tiedostosta
%% thesistemplate.tex.
%\documentclass[english,12pt,a4paper,pdftex,elec,utf8]{aaltothesis}
%\documentclass[english,12pt,a4paper,dvips]{aaltothesis}

\usepackage{graphicx}

%% Matematiikan fontteja, symboleja ja muotoiluja lisää, näitä tarvitaan usein 
\usepackage{amsfonts,amssymb,amsbsy}

%% Jos et jostain syystä pidä, miten alla oleva hyperref-paketti käyttää
%% fontteja, värejä yms., käytä tämän paketin makroja muuttamaan
%% fonttimäärittelyt. Katso paketin dokumentaatiota. Paketti määrittelee
%% \url-makron, joten ota paketti käyttöön, jos et käytä hyperref-pakettia.
%%
%\usepackage{url}

%% Saat pdf-tiedoston viittaukset ja linkit kuntoon seuraavalla paketilla.
%% Paketti toimii erityisen hyvin pdflatexin kanssa. 
%%
\usepackage{hyperref}
\hypersetup{pdfpagemode=UseNone, pdfstartview=FitH,
    colorlinks=true,urlcolor=red,linkcolor=blue,citecolor=black,
    pdftitle={Kuormantasaus www-palvelimissa},pdfauthor={Onni Lampi},
pdfkeywords={load balancing dns}}


%% Kaikki mikä paperille tulostuu, on tämän jälkeen
\begin{document}

%% Korjaa vastaamaan korkeakouluasi, jos automaattisesti asetettu nimi on 
%% virheellinen 
%%
%% Change the school field to specify your school if the automatically 
%% set name is wrong
% \university{aalto-yliopisto}
% \school{Sähkötekniikan korkeakoulu}

%% Vain kandityölle: Korjaa seuraavat vastaamaan koulutusohjelmaasi
%%
\degreeprogram{Informaatioteknologia}
%%

%% VAIN DI/M.Sc.- JA LISENSIAATINTYÖLLE: valitse laitos, 
%% professuuri ja sen professuurikoodi. 
%%
%\department{Radiotieteen ja -tekniikan laitos}
%\professorship{Piiriteoria}
%%

%% Valitse yksi näistä kolmesta
%%
\univdegree{BSc}
%\univdegree{MSc}
%\univdegree{Lic}

%% Oma nimi
%%
\author{Onni Lampi}

%% Opinnäytteen otsikko tulee tähän ja uudelleen englannin- tai 
%% ruostinkielisen abstraktin yhteydessä. Älä tavuta otsikkoa ja
%% vältä liian pitkää otsikkotekstiä. Jos latex ryhmittelee otsikon
%% huonosti, voit joutua pakottamaan rivinvaihdon \\ kontrollimerkillä.
%% Muista että otsikkoja ei tavuteta! 
%% Jos otsikossa on ja-sana, se ei jää rivin viimeiseksi sanaksi 
%% vaan aloittaa uuden rivin.
%% 
\thesistitle{Työnimi: Kuormantasaus www-palvelimissa}

\place{Espoo}

%% Kandidaatintyön päivämäärä on sen esityspäivämäärä! 
%% 
\date{14.5.2017}

%% Kandidaattiseminaarin vastuuopettaja tai diplomityön valvoja.
%% Huomaa tittelissä "\" -merkki pisteen jälkeen, ennen välilyöntiä ja
%% seuraavaa merkkijonoa. 
%% Näin tehdään, koska kyseessä ei ole lauseen loppu, jonka jälkeen tulee 
%% hieman pidempi väli vaan halutaan tavallinen väli.
%%
\supervisor{DI Mika Nupponen} %{Prof.\ Pirjo Professori}

%% Kandidaatintyön ohjaaja(t) tai diplomityön ohjaaja(t). Ohjaajia saa
%% olla korkeintaan kaksi.
%% 
%\advisor{Prof.\ Pirjo Professori}
\advisor{TkT Sebastian Sonntag, sebastian.sonntag@aalto.fi, COMNET}
%\advisor{DI Tina Tutkija}

%% Aaltologo: syntaksi:
%% \uselogo{aaltoRed|aaltoBlue|aaltoYellow|aaltoGray|aaltoGrayScale}{?|!|''}
%% Logon kieli on sama kuin dokumentin kieli
%%
\uselogo{aaltoRed}{''}

%% Tehdään kansilehti
%%
\makecoverpage


%% Suomenkielinen tiivistelmä
%% Kaikki tiivistelmässä tarvittava tieto (nimesi, työnnimi, jne.) käytetään
%% niin kuin se on yllä määritelty.
%% Tiivistelmän avainsanat
%%
\keywords{www palvelin kuormantasaus dns}
%% Tiivistelmän tekstiosa
%\begin{abstractpage}[finnish]
%    Tiivistelmää laajennetaan kandityön edistyessä  
%\end{abstractpage}

%% Pakotetaan uusi sivu varmuuden vuoksi, jotta 
%% mahdollinen suomenkielinen ja englanninkielinen tiivistelmä
%% eivät tule vahingossakaan samalle sivulle
%%
\newpage
%
%% Opinnäytteen ostikko englanniksi. Poista, jos et tarvitse sitä.
%\thesistitle{Thesis template}
%%\supervisor{Prof.\ Pirjo Professori}
%\advisor{D.Sc.\ (Tech.) Olli Ohjaaja}
%%\advisor{M.Sc.\ Polli Pohjaaja}
%\degreeprogram{Electronics and electrical engineering}
%\department{Department of Radio Science and Technology}
%\professorship{Circuit theory}
%%% Abstract keywords
%\keywords{Resistor, Resistance,\\ Temperature}
%%% Abstract text
%\begin{abstractpage}[english]
%Your abstract in English. Try to keep the abstract short, approximately 
% 100 words should be enough. Abstract explains your research topic, 
% the methods you have used, and the results you obtained.  
%\end{abstractpage}
%
%%% Force new page so that the Swedish abstract starts from a new page
%\newpage
%
%% Ruotsinkiellinen tiivitelmä. Poista, jos et tarvitse sitä.
%% 
%% Opinnäytteen ostikko ruotsiksi.
%\thesistitle{Arbetets titel}
%%\supervisor{Prof.\ Pirjo Professori}
%\advisor{TkD\ Olli Ohjaaja} %
%%\advisor{M.Sc.\ Tina Tutkija}
%\degreeprogram{Elektronik och elektroteknik}
%\department{Institutionen för radiovetenskap och -teknik}%
%\professorship{Kretsteori}  %
%%% Abstract keywords
%\keywords{Nyckelord p\aa{} svenska,\\ Temperatur}
%%% Abstract text
%\begin{abstractpage}[swedish]
% Sammandrag p\aa{} svenska.
% Try to keep the abstract short, approximately 
% 100 words should be enough. Abstract explains your research topic, 
% the methods you have used, and the results you obtained.  
%\end{abstractpage}

%% Note that if you are writting your master's thesis in English, place
%% the English abstract first followed by the possible Finnish abstract

%% Esipuhe 
%%
%\mysection{Esipuhe}
%
%Kirjoitetaan myöhemmin, mielellään viimeisenä osana työtä.\\
%
%\vspace{5cm}
%Otaniemi, 19.6.2015
%
%\vspace{5mm}
%{\hfill Onni S.\ Lampi \hspace{1cm}}
%
%%% Pakotetaan varmuuden vuoksi esipuheen jälkeinen osa
%%% alkamaan uudelta sivulta
%\newpage


%% Sisällysluettelo
\thesistableofcontents


%% Symbolit ja lyhenteet
\mysection{Lyhenteet ja termistö}

\subsection*{Lyhenteet}

\begin{tabular}{ll}
    DNS & Domain Name Service \\
    TCP & Transmission Control Protocol\\
\end{tabular}
\subsection*{Termistöä}

\begin{tabular}{ll}
    Kuormantasaus & palvelinpyyntöjen hajauttaminen esimerkiksi\\
                  & maantieteellisen sijainnin perusteella.\\
\end{tabular}


%% Sivulaskurin viilausta opinnäytteen vaatimusten mukaan:
%% Aloitetaan sivunumerointi arabialaisilla numeroilla (ja jätetään
%% leipätekstin ensimmäinen sivu tyhjäksi, 
%% ks. alla \thispagestyle{empty}).
%% Pakotetaan lisäksi ensimmäinen varsinainen tekstisivu alkamaan 
%% uudelta sivulta clearpage-komennolla. 
%% clearpage on melkein samanlainen kuin newpage, mutta 
%% flushaa myös LaTeX:n floatit 
%% 
\cleardoublepage
\storeinipagenumber
\pagenumbering{arabic}
\setcounter{page}{1}


%% Leipäteksti alkaa
%%
%\chapter{Suunnitelma}
\section{Johdanto}
Internetin alkuajoista lähtien vuosikymmenten ajan vallitsi tarjotuissa palveluissa selkeä trendi: yksi palvelu, yksi palvelin.
Palvelinten resurssit ja pyyntoihin vastaamiskyky eivät kuitenkaan skaalaudu lineaarisesti, vaan vastaan tulee usein jo pelkkä kaistanleveys.
Internetin laajentuessa ja kasvaessa nousee hajautettujen palvelinratkaisujen merkitys yhä suuremmaksi.
Mitä enemmän edellä mainittuja ratkaisuja käytetään tuotantoympäristöissä, sitä merkityksellisempää on kuormantasaus.
Kuormantasauksella voidaan saavuttaa merkittäviä ja konkreettisia etuja kaistanleveyden ja palvelinkapasiteetin säästämisessä.\\

Kuormantasausta voidaan harjoittaa monin eri tavoin, toisistaan jopa merkittävästi eroavilla tekniikoilla.
Tässä työssä käsitellään ja vertaillaan erilaisia kuormantasaustenkiikoita, niiden skaalautuvuutta sekä muita ominaisuuksia.
Työssä keskitytään muutamaan jo olemassa olevaan ja laajasti käytettiin teknologiaan.
Lisäksi käsitellään nopeasti mahdollisesti uusia ja kokeellisia teknologioita.

\newpage

\section{Työn tavoite}
Tämän työn tavoitteena on perehtyä ja oppia ymmärtämään erilaisia kuormantasaustekniikoita, sekä selventää kirjallisuudessa käsitellyt merkittävät erot erilaisten teknologioiden välillä.
Keskiössä on myös selvitys siitä, miksi erilaisissa tuotantoympäristöissä käytetään tiettyjä teknologioita.
Tarkastelun ei ole tarkoitus pureutua jokaisen teknologian historiaan ja totuttuihin käytänteisiin, vaan tutkia aihetta juurikin käyttöasteen kannalta.\\

Työssäni on alustavasti kolme keskeistä tutkimuskysymystä:
\begin{itemize}
    \item Mitkä ovat merkittävimmät kuormantasausteknologiat?
    \item Onko olemassa vain yhtä oikeaa tapaa toteuttaa kuormantasaus?
    \item Miten laajalti kuormantasausta käytetään ja missä yhteyksissä?
\end{itemize}
Työn tavoitteena on siis saada hyvä kokonaiskuva teknologioiden nykytilasta.
Tavoitteena on myös tehdä itse kandityö englanniksi.

\newpage

\section{Teoria, aineisto, menetelmä}

Työn yleisenä teoriataustana toimii kandidaattivaiheessa saatu perusymmärrys tietoverkkojen rakenteesta laajalla ja pienellä skaalalla.
On tärkeää ymmärtää ennakkoon perusteet Internetin toiminnasta, sekä tuntea niiden historiaa siinä määrin kuin se on käyttöasteen kannalta olennaista.
On perusteltua todeta, että vahva teoriatausta työn suorittamiseen on olemassa ja sitä tullaan työssä varmasti hyödyntämään.

Työssä ei tulla juurikaan suorittamaan empiiristä tutkimusta, vaan painopiste on kirjallisuudessa.
Tämä johtuu siitä, ettei kyseessä olevaa teknologiaa ole mielekästä kokeilla laboratorioympäristössä; sitä tehdään tarpeeksi jo olemassa olevien tutkimusten taustaksi.
Tietoa työn tueksi on helppo etsiä niin perinteisestä kirjallisuudesta, kuin tutkimuspapereista.

Materiaali on usein huomattavan mielipidepainotteista.
Tämän takia on tärkeää osata löytää matariaalista tekniset seikat ja teknologian ydin, ei niinkään keskittyä kirjoittajan tai kirjoittajien mielipiteisiin kyseisestä teknologiasta.

Omien johtopäätösten osallisuus tulee olemaan työn loppupuolella suuressa roolissa, vaikkakin kyse on enemmän asioiden tulkitsemisesta kuin suurista tieteellisistä johtopäätöksistä.

\newpage
\section{Aikataulu kesälle 2017}
Alla aikataulua viikkotarkkuudella työn etenemisestä:
\begin{itemize}
    \item \textbf{Viikko 20:} Luentoviikko. Esitehtävän deadline maanantaina. Aineiston keräämistä ja työkaluihin tutustumista.
    \item \textbf{Viikko 21:} Materiaalin keräämistä.
    \item \textbf{Viikko 22:} Materiaalin keräämistä ja jäsentelyä. Teoriaosuuden ranka valmiiksi. Tiedonhaun harjoitus
    \item \textbf{Viikko 23:} Materiaalin keräämistä ja jäsentelyä. Vedos 1 valmiiksi.
    \item \textbf{Viikko 24:} Lähdeluettelon viimeistelyä.
    \item \textbf{Viikko 25:} Työn hiomista ohjaajan palautteen perusteella.
    \item \textbf{Viikko 26:} Toisen vedoksen viimeistelyä.
    \item \textbf{Viikko 27:} Perjantaina toisen vedoksen DL.
    \item \textbf{Viikko 28:} Pienryhmäkeskustelua.
    \item \textbf{Viikko 29:} Pienryhmäkeskustelua ja tämän raportointia. Kolmannen vedoksen viimeistelyä.
    \item \textbf{Viikko 30:} Kolmas vedos palautettava maanantaina.
    \item \textbf{Viikko 31:} Pienryhmäkeskustelua ja tämän raportointia. Neljännen vedoksen viimeistelyä.
    \item \textbf{Viikko 32:} Hiomista
    \item \textbf{Viikko 33:} Perjantaihin mennessä kommentit ohjaajalta.
    \item \textbf{Viikko 34:} Sunnuntaina semmakalvojen DL.
    \item \textbf{Viikko 35:} Seminaariviikko.
    \item \textbf{Viikko 38:} Viimeisen version DL
\end{itemize}
Tavoitteena on saada valtaosa työn tekstistä valmiiksi ennen kolmatta vedosta.
Tämän jälkeen keskityn kielelliseen asuun, sekä seminaarikalvoihin hiomiseen.
\clearpage

\nocite{*}
\bibliography{kandi}
\bibliographystyle{ieeetr}

\end{document}
