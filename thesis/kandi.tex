%% Käytä toinen näistä:
%% ensimmäinen, jos käytät pdflatexia, joka kääntää tekstin suoraan 
%% pdf-tiedostoksi (kuvat on oltava jpg- tai pdf-tiedostoina)
%% toinen, jos haluat tuottaa ps-tiedostoa (käytä eps-formaattia kuville,
%% alä käytä ps-muotoisia kuvia!)
%%
%\documentclass[english,12pt,a4paper,elec,utf8]{aaltothesis}
%\documentclass[finnish,12pt,a4paper,dvips]{aaltothesis}

%% Käytä näitä, jos kirjoitat englanniksi. Katso englanninokset tiedostosta
%% thesistemplate.tex.
\documentclass[english,12pt,a4paper,pdftex,elec,utf8]{aaltothesis}
%\documentclass[english,12pt,a4paper,dvips]{aaltothesis}

\usepackage{graphicx}

%% Matematiikan fontteja, symboleja ja muotoiluja lisää, näitä tarvitaan usein 
\usepackage{amsfonts,amssymb,amsbsy}

%% Jos et jostain syystä pidä, miten alla oleva hyperref-paketti käyttää
%% fontteja, värejä yms., käytä tämän paketin makroja muuttamaan
%% fonttimäärittelyt. Katso paketin dokumentaatiota. Paketti määrittelee
%% \url-makron, joten ota paketti käyttöön, jos et käytä hyperref-pakettia.
%%
%\usepackage{url}

%% Saat pdf-tiedoston viittaukset ja linkit kuntoon seuraavalla paketilla.
%% Paketti toimii erityisen hyvin pdflatexin kanssa. 
%%
\usepackage{hyperref}
\hypersetup{pdfpagemode=UseNone, pdfstartview=FitH,
    colorlinks=true,urlcolor=red,linkcolor=blue,citecolor=black,
    pdftitle={WWW load balancing},pdfauthor={Onni Lampi},
pdfkeywords={load balancing dns}}


%% Kaikki mikä paperille tulostuu, on tämän jälkeen
\begin{document}

%% Korjaa vastaamaan korkeakouluasi, jos automaattisesti asetettu nimi on 
%% virheellinen 
%%
%% Change the school field to specify your school if the automatically 
%% set name is wrong
% \university{aalto-yliopisto}
% \school{Sähkötekniikan korkeakoulu}

%% Vain kandityölle: Korjaa seuraavat vastaamaan koulutusohjelmaasi
%%
\degreeprogram{Information technology}
%%

%% VAIN DI/M.Sc.- JA LISENSIAATINTYÖLLE: valitse laitos, 
%% professuuri ja sen professuurikoodi. 
%%
%\department{Radiotieteen ja -tekniikan laitos}
%\professorship{Piiriteoria}
%%

%% Valitse yksi näistä kolmesta
%%
\univdegree{BSc}
%\univdegree{MSc}
%\univdegree{Lic}

%% Oma nimi
%%
\author{Onni Lampi}

%% Opinnäytteen otsikko tulee tähän ja uudelleen englannin- tai 
%% ruostinkielisen abstraktin yhteydessä. Älä tavuta otsikkoa ja
%% vältä liian pitkää otsikkotekstiä. Jos latex ryhmittelee otsikon
%% huonosti, voit joutua pakottamaan rivinvaihdon \\ kontrollimerkillä.
%% Muista että otsikkoja ei tavuteta! 
%% Jos otsikossa on ja-sana, se ei jää rivin viimeiseksi sanaksi 
%% vaan aloittaa uuden rivin.
%% 
\thesistitle{WWW load balancing}

\place{Espoo}

%% Kandidaatintyön päivämäärä on sen esityspäivämäärä! 
%% 
\date{12.6.2017}

%% Kandidaattiseminaarin vastuuopettaja tai diplomityön valvoja.
%% Huomaa tittelissä "\" -merkki pisteen jälkeen, ennen välilyöntiä ja
%% seuraavaa merkkijonoa. 
%% Näin tehdään, koska kyseessä ei ole lauseen loppu, jonka jälkeen tulee 
%% hieman pidempi väli vaan halutaan tavallinen väli.
%%
\supervisor{DI Mika Nupponen} %{Prof.\ Pirjo Professori}

%% Kandidaatintyön ohjaaja(t) tai diplomityön ohjaaja(t). Ohjaajia saa
%% olla korkeintaan kaksi.
%% 
%\advisor{Prof.\ Pirjo Professori}
\advisor{TkT Sebastian Sonntag, sebastian.sonntag@aalto.fi, COMNET}
%\advisor{DI Tina Tutkija}

%% Aaltologo: syntaksi:
%% \uselogo{aaltoRed|aaltoBlue|aaltoYellow|aaltoGray|aaltoGrayScale}{?|!|''}
%% Logon kieli on sama kuin dokumentin kieli
%%
\uselogo{aaltoRed}{''}

%% Tehdään kansilehti
%%
\makecoverpage


%% Suomenkielinen tiivistelmä
%% Kaikki tiivistelmässä tarvittava tieto (nimesi, työnnimi, jne.) käytetään
%% niin kuin se on yllä määritelty.
%% Tiivistelmän avainsanat
%%
\thesistitle{Kuormantasaus www-palvelimissa}
\degreeprogram{Informaatioteknologia}
\department{Tietoliikenne- ja tietoverkkotekniikan laitos}
\keywords{www, palvelin, kuormantasaus, dns}
%% Tiivistelmän tekstiosa
\begin{abstractpage}[finnish]
    Tiivistelmää laajennetaan kandityön edistyessä  
\end{abstractpage}

%% Pakotetaan uusi sivu varmuuden vuoksi, jotta 
%% mahdollinen suomenkielinen ja englanninkielinen tiivistelmä
%% eivät tule vahingossakaan samalle sivulle
%%
\newpage
%
%% Opinnäytteen ostikko englanniksi. Poista, jos et tarvitse sitä.
\thesistitle{WWW load balancing}
%\supervisor{Prof.\ Pirjo Professori}
\advisor{D.Sc.\ (Tech.) Sebastian Sonntag}
%\advisor{M.Sc.\ Polli Pohjaaja}
\degreeprogram{Information technology}
\department{Department of Communications and Networking}
%% Abstract keywords
\keywords{www, server, load balancing, dns}
%% Abstract text
\begin{abstractpage}[english]
Abstarct will be written at a later time.
\end{abstractpage}
%
% Esipuhe 
%
\mysection{Preface}

Last thing to be written.\\

\vspace{5cm}
Otaniemi, XX.X.2017

\vspace{5mm}
{\hfill Onni S.\ Lampi \hspace{1cm}}

%% Pakotetaan varmuuden vuoksi esipuheen jälkeinen osa
%% alkamaan uudelta sivulta
\newpage


%% Sisällysluettelo
\thesistableofcontents


%% Symbolit ja lyhenteet
\mysection{Abbrevations and common terms}

\subsection*{Abbrevations}

\begin{tabular}{ll}
    DNS & Domain Name Service \\
    TCP & Transmission Control Protocol\\
\end{tabular}
\subsection*{Common terms}

\begin{tabular}{ll}
    Load balancing & Distribution of server requests\\
                  & by some method.\\
    GeoDNS        & A method to balance server load\\
                  & based on the geographical location.
\end{tabular}


%% Sivulaskurin viilausta opinnäytteen vaatimusten mukaan:
%% Aloitetaan sivunumerointi arabialaisilla numeroilla (ja jätetään
%% leipätekstin ensimmäinen sivu tyhjäksi, 
%% ks. alla \thispagestyle{empty}).
%% Pakotetaan lisäksi ensimmäinen varsinainen tekstisivu alkamaan 
%% uudelta sivulta clearpage-komennolla. 
%% clearpage on melkein samanlainen kuin newpage, mutta 
%% flushaa myös LaTeX:n floatit 
%% 
\cleardoublepage
\storeinipagenumber
\pagenumbering{arabic}
\setcounter{page}{1}


%% Leipäteksti alkaa
%%
%\chapter{Suunnitelma}
\section{Introduction}
For decades Internet was ruled by one simple rule; one service, one server.
As the amount of users and transferred date exponentially increases, so does the need for more servers.
However, the amount of servers is irrelevant if all clients only connect to one of them.
So the need for load balancing was born in the mid 90's.
Couple of years later, a rudimentary method was devised and specified that divided the load based on the geographical location of the client, GeoDNS.\\

Load balancing can be done on various layers on the network, it can be performed for the servers or network bandwidth, for example.
This thesis briefly goes through the different technigues and their use cases and concentrate on the couple most prominent ones.
New and experimental methods are also taken into account, but they are not the focus of this thesis, rather a curiosity.

\newpage

\section{Research questions}

Research questions are as follows
\begin{itemize}
    \item Define different load balancing methods and their usage.
    \item Is there one simple way of performing load balancing?
    \item How widely and why is load balancing used?
\end{itemize}
\newpage

\section{Theoretical background}
\subsection{A brief history lesson}
\subsection{Why load balancing?}

\newpage

\section{Different load balancing methods}
\subsection{Location based load balancing}
\subsection{Content-blind load balancing}
\subsection{Content-aware load balancing}
\newpage



\section{Analysis}
\clearpage

\nocite{*}
\bibliography{kandi}
\bibliographystyle{ieeetr}

\end{document}
